
\section{Discussion}
warm up period
Samplers could implement $\varepsilon$ schedule, potentially improving especially ULA
Perturbation variance schedule


\subsection{Framework}
During the course of this project the framework grew organically with some major milestones along the way being the creation of the multi-chain samplers, 
the base class for the energy models and the implementation of the recovery adapter and the training observers.

Many of the created components turned out well and reusable but there is a lot of room for extension and potential improvement in the pipeline.
As the issues and findings discussed in the results section show, the samplers could profit significantly from introducing an $\varepsilon$ schedule.
By setting the initial $\varepsilon$ high and decreasing it in later epochs, one could avoid the problem of steep gradients in the beginning,
and at the same time approximate the model distribution better in the later stages, when an overall burnin of the chains has taken place.

Another way to counteract the initially poor representation of the model distribution would be to introduce a warmup burnin, before the actual training procedure starts.
This would allow using even a poorly chosen start batch with lower burnin periods during the training.

Similarly it could make sense to introduce a schedule for the perturbation variance as well.


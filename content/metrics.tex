
\section{Metrics}

Recovery Likelihood promises an improvement in convergence speed by accelerating the calculation of the likelihood gradient.
Does this improvement come at the cost of quality, reliability or increased maintenance?
What would we like to have:
\begin{enumerate}
	\item Accurate parameter estimates
	\item A sufficiently close parametric representation/model for the data distribution
	\item Consistent quality in training
	\item Independence of the model
	\item Straightforward parameter choice and low dependency for perturbation 
\end{enumerate}

Comparing the likelihood approaches and the involved samplers in a rigorous way requires some measure 
of both the quality of the final estimates and the reliability of the approach.


\subsection{Metrics for Estimated Parameters}

Many metrics could be used to evaluate the quality of the estimates, but for simplicity, 
here only some simple metrics of the distance between the known parameters and the estimated parameters are employed.
For vector valued parameters $\bm{\theta} \in \mathbb{R}^m$ the $p$-norm family, and specifically the euclidian norm is used i.e.
\[
	\text{error}_p (\bm{\theta}, \hat{\bm{\theta}}) = \norm[p]{ \bm{\theta} - \hat{\bm{\theta}} } = \left(\sum_{i=1}^m (\theta_i - \hat{\theta}_i)^p \right)^{ \frac{1}{p} }
\]

while for matrix valued parameters, like e.g. the covariance matrices of the Gaussian model, the Frobenius norm of the difference is applied
\[
	\text{error} (\bm{\theta}, \hat{\bm{\theta}}) = \norm[F]{ \bm{\theta} - \hat{\bm{\theta}} } 
	= \left(\sum_{i=1}^m \sum_{j=1}^n (\theta_{i j} - \hat{\theta}_{i j})^2 \right)^{ \frac{1}{2} }
\]

These give a measure of how good the respective approach is at estimating the parameters accurately and would be specifically important in contexts, 
where the parameters have a specific meaning and the estimates are meant to be interpretable.







\begin{comment}
%------------------------------------------------------------------------------------------------------------------------------------------------------------------------
\subsection{Metrics for Estimated Model}
On the other hand we can judge the quality of the estimates based on how closely the parametrised distribution resembles the target distribution.
This approach is particularly useful if there is no assumption that the underlying distribution can be represented by a parametric model, 
or when the actual parameter estimates are less important than the practical use of the model.
Especially for neural network models this is the appropriate approach as typically the weights themselves have no meaning 
and drastically different weight configurations can produce similar distributions.

Wasserstein metric.
Let $(M,d)$ be a metric space
\[
	W_p(\mu, \nu) = \left( \inf_{\gamma \in \Gamma(\mu, \nu)} \int_{M \times M} d(x, y) \dif \gamma(x,y) \right)^{ \frac{1}{p} }
\]

Kullback–Leibler divergence / Relative Entropy / I divergence

More generally, if $P$ and $Q$ are probability measures on a measurable space $M$ and $P$ is absolutely continuous with respect to $Q$, 
then the relative entropy from $Q$ to $P$ is defined as
\[
	D_{KL} (P || Q) = \int_M \log \frac{ P(\dif x) }{ Q(\dif x) } Q(\dif x)
\]


Our Context:
\[
	D_{KL} (p || p_{\bm{\theta}}) = \int_{\mathbb{R}^d} p(\bm{x}) \log \frac{ p(\bm{x}) }{ p_{\bm{\theta}}(\bm{x}) } \dif \bm{x} 
	= \mathbb{E} \log \frac{ p(\bm{X}) }{ p_{\bm{\theta}}(\bm{X}) }
\]



\subsection{Metrics for Estimator}
Apart from the quality specific estimates the reliability of the approach is also of key importance.
I.e. how consistent and stable is the estimate during multiple training runs.
In order to see this we can look at the variance of the estimates.
\end{comment}
























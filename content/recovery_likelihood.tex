\section{Recovery Likelihood}

\begin{comment}
Let $\bm{x} \sim p_{\text{data}}(\bm{x})$ denote a training example, and $p_{\bm{\theta}}(\bm{x})$ denote a model’s probability density function 
that aims to approximates $p_{\text{data}}(\bm{x})$. 
An energy-based model (EBM) is defined as:

\[
	p_{\bm{\theta}}(\bm{x}) = \frac{1}{Z_{\bm{\theta}}} \exp( - U_{\bm{\theta}} (\bm{x}) )
\]

where $Z_{\bm{\theta}} = \int \exp( - U_{\bm{\theta}} (\bm{x}) ) d\bm{x}$ is the partition function which is analytically intractable for high dimensional $\bm{x}$.
\end{comment}

The idea of recovery likelihood is to create perturbed samples $\tilde{\bm{x}} = \bm{x} + \sigma \varepsilon$ with $\varepsilon \sim \mathcal{N}(0, I)$, 
from our original dataset and then use the conditional distribution
\[
\begin{aligned}
	p_{\bm{\theta}}(\bm{x} | \tilde{\bm{x}}) &=  \frac{ p(\tilde{\bm{x}} | \bm{x})  p_{\bm{\theta}}(\bm{x}) }{ p(\tilde{\bm{x}}) } \\
	&= \frac{ \exp( -\frac{ \norm{\tilde{\bm{x}} - \bm{x}}^2 }{ 2 \sigma^2 })  \exp( - U_{\bm{\theta}} (\bm{x}) ) }{ Z_{\bm{\theta}} (2\pi \sigma^2)^{ \frac{n}{2} } p(\tilde{\bm{x}}) } \\
	&= \frac{ \exp( - U_{\bm{\theta}} (\bm{x}) - \frac{ \norm{\tilde{\bm{x}} - \bm{x}}^2 }{ 2 \sigma^2 }) }{ Z_{\bm{\theta}} (2\pi \sigma^2)^{ \frac{n}{2} } p(\tilde{\bm{x}}) } \\
	&= \frac{ \exp( -U_{\bm{\theta}} (\bm{x}) - \frac{ \norm{\tilde{\bm{x}} - \bm{x}}^2 }{ 2 \sigma^2 }) }{ \tilde{Z}_{\bm{\theta}} (\tilde{\bm{x}}) } \\
\end{aligned}
\]

Suppose now that for our samples $(\bm{x}_i )_{i=1}^n $ we have the perturbed samples $(\tilde{\bm{x}}_i )_{i=1}^n $ that we gained by adding noise with
$\tilde{\bm{x}}_i = \bm{x}_i + \sigma \varepsilon_i$, then the normalised recovery-log-likelihood (RLL) is defined as

\[
	\frac{1}{n} \mathcal{J}(\bm{\theta}) = \frac{1}{n} \sum_{i=1}^n \log p_{\bm{\theta}}(\bm{x} | \tilde{\bm{x}})
\]

With the expression above this is

\[
	\frac{1}{n} \sum_{i=1}^n \log p_{\bm{\theta}}(\bm{x} | \tilde{\bm{x}}) 
	= \frac{1}{n} \sum_{i=1}^n \left ( - U_{\bm{\theta}} (\bm{x}) - \frac{ \norm{\tilde{\bm{x}} - \bm{x}}^2 }{ 2 \sigma^2 }) - \log(\tilde{Z}_{\bm{\theta}} (\tilde{\bm{x}}) \right)
\]
where
\[
\begin{aligned}
	\log(\tilde{Z}_{\bm{\theta}} (\tilde{\bm{x}}) )
	&= \log( Z_{\bm{\theta}} (2\pi \sigma^2)^{ \frac{n}{2} } p(\tilde{\bm{x}})) \\
	&= \log( Z_{\bm{\theta}}) + \frac{n}{2}\log(2\pi \sigma^2) + \log( p(\tilde{\bm{x}})) \\
\end{aligned}
\]
Thus upon differentiating by $\theta$ we have

\[
\begin{aligned}
	\frac{\dif}{\dif \bm{\theta} } \frac{1}{n} \mathcal{J}(\bm{\theta})
	&= - \frac{1}{n} \sum_{i=1}^n \frac{\dif}{\dif \bm{\theta} } U_{\bm{\theta}} (\bm{x}) - \frac{\dif}{\dif \bm{\theta} } \log(\tilde{Z}_{\bm{\theta}} (\tilde{\bm{x}}) \\
	&= - \frac{1}{n} \sum_{i=1}^n \frac{\dif}{\dif \bm{\theta} } U_{\bm{\theta}} (\bm{x}) 
		- \frac{\dif}{\dif \bm{\theta} } ( \log( Z_{\bm{\theta}}) + \frac{n}{2}\log(2\pi \sigma^2) + \log( p(\tilde{\bm{x}})) ) \\
	&= - \frac{1}{n} \sum_{i=1}^n \frac{\dif}{\dif \bm{\theta} } U_{\bm{\theta}} (\bm{x}) - \frac{\dif}{\dif \bm{\theta} } \log( Z_{\bm{\theta}})  \\
	&= \frac{\dif}{\dif \bm{\theta} } \frac{1}{n} \mathcal{L}(\bm{\theta})  \\
\end{aligned}
\]

So the gradients of RLL and standard LL coincide. 
This means we can sample from the conditional distribution $p_{\bm{\theta}}(\bm{x} | \tilde{\bm{x}})$, which is less multimodal.
Indeed Gao and Song have shown that for small $\sigma$ the conditional can be approximated by a normal distribution,
\[
	p_{\bm{\theta}}(\bm{x} | \tilde{\bm{x}}) \doteq \mathcal{N} \left( \bm{x} | \tilde{\bm{x}} - \sigma^2 \nabla_{\bm{x}} U_{\bm{\theta}} (\tilde{\bm{x}}) , \sigma^2 I \right)
\]

so the smaller we choose $\sigma$, the more the conditional resembles a shifted normal distribution, which is much easier to sample from.
When we use the previously introduced Markov chain samplers we can treat the perturbed distribution like another EBM model with conditional energy functional
\[
	\tilde{U}_{\bm{\theta}} (\bm{x}) = U_{\bm{\theta}} (\bm{x}) + \frac{ \norm{\tilde{\bm{x}} - \bm{x}}^2 }{ 2 \sigma^2 }
\]

and use its gradient to generate the samples. This gradient is given by
\[
\begin{aligned}
	\nabla_{\bm{x}} \tilde{U}_{\bm{\theta}} (\bm{x}) 
	&= \nabla_{\bm{x}} (U_{\bm{\theta}} (\bm{x}) + \frac{ \norm{\tilde{\bm{x}} - \bm{x}}^2 }{ 2 \sigma^2 }) \\
	&= \nabla_{\bm{x}} U_{\bm{\theta}} (\bm{x}) + \nabla_{\bm{x}} \frac{ (\tilde{\bm{x}} - \bm{x})^T (\tilde{\bm{x}} - \bm{x}) }{ 2 \sigma^2 } \\
	&= \nabla_{\bm{x}} U_{\bm{\theta}} (\bm{x}) - \frac{ \tilde{\bm{x}} - \bm{x} }{ \sigma^2 } \\
\end{aligned}
\]


\begin{comment}
\subsection{Normal Approximation to Conditional}

The previous discussion revolved around making the model distribution easier to sample from by essentially convolving the data with a Gaussian 
and then sampling from the conditional. 
This conditional is less multimodal and the hope is that the Markov chain samplers converge to the target faster and allow to create representative samples with less iterations.

However, when the perturbation variance $\sigma$ is sufficiently small, 
one can obtain samples more directly by sampling from a normal approximation to the conditional.
When we do a first-order Tailor approximation to the conditional energy functional at $\tilde{\bm{x}}$ we get

\[
\begin{aligned}
	\tilde{U}_{\bm{\theta}} (\bm{x}) 
	&\doteq \tilde{U}_{\bm{\theta}} (\tilde{\bm{x}}) + (\bm{x} - \tilde{\bm{x}})^T  \nabla_{\bm{x}} \tilde{U}_{\bm{\theta}} (\tilde{\bm{x}}) 
		+ \mathcal{O}(\norm{\bm{x} - \tilde{\bm{x}}}^2) \\
	&= U_{\bm{\theta}} (\tilde{\bm{x}}) + \frac{ (\tilde{\bm{x}} - \tilde{\bm{x}})^T (\tilde{\bm{x}} - \tilde{\bm{x}}) }{ 2 \sigma^2 }
		+ (\bm{x} - \tilde{\bm{x}})^T (\nabla_{\bm{x}} U_{\bm{\theta}} (\tilde{\bm{x}}) - \frac{ \tilde{\bm{x}} - \tilde{\bm{x}} }{ \sigma^2 }) 
		+ \mathcal{O}(\norm{\bm{x} - \tilde{\bm{x}}}^2)\\
	&= U_{\bm{\theta}} (\tilde{\bm{x}}) + (\bm{x} - \tilde{\bm{x}})^T \nabla_{\bm{x}} U_{\bm{\theta}} (\tilde{\bm{x}}) + \mathcal{O}(\norm{\bm{x} - \tilde{\bm{x}}}^2)\\
\end{aligned}
\]



\[
\begin{aligned}
	\tilde{U}_{\bm{\theta}} (\bm{x}) 
	&= U_{\bm{\theta}} (\bm{x}) + \frac{ \norm{\tilde{\bm{x}} - \bm{x}}^2 }{ 2 \sigma^2 }) \\
	&\doteq U_{\bm{\theta}} (\tilde{\bm{x}}) + \frac{ (\tilde{\bm{x}} - \tilde{\bm{x}})^T (\tilde{\bm{x}} - \tilde{\bm{x}}) }{ 2 \sigma^2 } 
		+ \nabla_{\bm{x}} \\
\end{aligned}
\]
\end{comment}
































